\documentclass[12pt]{article}
\usepackage[margin=0.8in]{geometry} 
\usepackage{amsmath,amsthm,amssymb,amsfonts}
\usepackage{graphicx}
\usepackage{algorithmic}
\usepackage{dsfont}
\usepackage{geometry}
\usepackage[toc,page]{appendix}
\usepackage{caption}
\usepackage{subcaption}
\usepackage[utf8]{inputenc}

\newcommand{\N}{\mathbb{N}}
\newcommand{\Z}{\mathbb{Z}}

 
\newenvironment{problem}[2][Problem]{\begin{trivlist}
\item[\hskip \labelsep {\bfseries #1}\hskip \labelsep {\bfseries #2.}]}{\end{trivlist}}
%If you want to title your bold things something different just make another thing exactly like this but replace "problem" with the name of the thing you want, like theorem or lemma or whatever
 
\begin{document}
\title{Pricing Financial Derivatives   \\ Project for Evaluation 2 - February 20, 2017}
\author{Angus McKay, Laura Roman, Euan Dowers, Veronika Kyuchukova}
\date{}
\maketitle

\title {{\bf Exercise 1:} {\bf Black-Scholes model }}
{\bf Consider the Black-Scholes model:}
\begin{align*}
& dX_t = 2X_t dt + 1.5X_tdB_t, && X_0=1
\end{align*}
{\bf (i) Write its solution \\
(ii) Generate on an interval a Brownian path. Using that path plot a trajectory of the solution. Using the same Brownian path use the Euler scheme approximation to plot another trajectory of the equation. Compare these plots with the simulation of the fact solution.} \\

We can rewrite the differential equation $dX_t = 2X_t dt + 1.5X_tdB_t$ as follows:
\begin{equation}
\frac{dX_t}{X_t} = 2dt + 1.5dB_t
\end{equation}
So we can integrate as follows:
\begin{equation} \label{uno}
\int_0^t \frac{dX_s}{X_s} = \int_0^t2ds + \int_0^t 1.5dB_s = 2t + 1.5B_t
\end{equation}
Now let
\begin{equation}
f(t,X_t)= f(X_t)= ln(X_t).
\end{equation}
Its partial derivatives are therefore:
\begin{align}
 \frac{\partial f(x,t)}{\partial t}= 0 &&
 \frac{\partial f(t,x)}{\partial x} = \frac{1}{x}&&
 \frac{\partial^2f(x,t)}{\partial x^2} = \frac{-1}{x^2}  = 0 
\end{align}

Recall Ito's formula:
\begin{align}
df(t,x) = d(lnX_t) =& \frac{\partial f(x,t)}{\partial t} dt + \frac{\partial f(t,x)}{\partial x}dx + \frac{1}{2}\frac{\partial^2f(x,t)}{\partial x^2}(dx)^2= \\
= & \frac{1}{X_t}dX_t +\frac{1}{2}\frac{-1}{X_t^2}(dX_t)^2  
\end{align}
Where $dX_t^2$ can be computed by using the differential equation provided:
\begin{align}
(dX_t)^2   = (2X_tdt+1.5X_tdB_t)^2 = 4X_t^2(dt)^2 + (1.5X_t)^2(dB_t)^2+2\cdot2\cdot1.5X_t^2dtdB_t
\end{align}
From Ito's calculus we do know: $(dt)^2=0; (dB_t)^2=dt$ and $dtdB_t=0$. Therefore
\begin{align}
(dX_t)^2   = (1.5)^2X_t^2dt
\end{align}
and so
\begin{equation}
d(lnX) = \frac{dX_t}{X_t} - \frac{1}{2}\frac{-1}{X_t^2}(1.5 \cdot X_t)^2 dt  .
\end{equation}
We can now integrate this expression:
\begin{equation}
\int_0^t d(lnX) = \int_0^t \frac{dX_s}{X_s} - \int_0^t \frac{-1.5^2}{2} ds   
\end{equation}
\begin{equation}\label{dos}
\Rightarrow \ln \frac{X_t}{X_0} =  \int_0^t \frac{dX_s}{X_s} -  \frac{-1.5^2}{2} t
\end{equation}
Substituting the term $\int_0^t \frac{dX_s}{X_s}$ from equation (\ref{dos}) with the expression given in equation (\ref{uno}) gives
\begin{equation}
\ln \frac{X_t}{X_0} = 1.5B_t + \bigg(2-\frac{1.5^2}{2}\bigg)t
\end{equation}
Therefore
\begin{equation}
X_t = X_0 \exp\bigg(1.5B_t+0.875\cdot t\bigg)
\end{equation}
Recall $X_0=1$;
\begin{equation}
\boxed{X_t = \exp\bigg(1.5B_t+0.875\cdot t\bigg)}
\end{equation}
\title {{\bf Exercise 2:} {\bf Vasiceck model for interest rates. }}
{\bf Consider the SDE}
\begin{align*}
&dX_t = a(m-X_t)dt + \sigma dB_t; && X_0=x
\end{align*}
{\bf where $a>0, \sigma >0$ and $m\ge 0$. This process tends to drift forward its long-term mean $m$: such a process is called mean-reverting. 
(i) {\bf Differentiating $e^{at}X_t$, find the solution to this SDE.}
(ii) {\bf Compute $\mathbf{E}[X_t]$} and $Var(X_t)$. Take the limit of the expectation as $t\longrightarrow \infty$ to observe the mean-reverting phenomena.}
(iii) {\bf Using the Euler's scheme on an interval simulate several trajectories of the process for different values of $a,\sigma, m$. Do you observe the mean-reverting phenomena? } \\

(i)We are given the following $f(t,X_t)$ function:
\begin{equation}
f(t,X_t)= e^{at}X_t
\end{equation}
We can compute its derivative by:
\begin{equation} \label{f}
d(f(t,X_t)) = d(e^{at}X_t) = e^{at}X_t+X_t\cdot a e^{at}dt = e^{at} \bigg(dX_t + aX_t dt \bigg)
\end{equation}
On the other hand, we are considering the SDE:
\begin{align}
dX_t &= a(m-X_t)dt + \sigma dB_t \\
\label{sde}
\Rightarrow dX_t + aX_tdt &= amdt+\sigma dB_t.
\end{align}
The left-hand side of equation (\ref{sde}) multiplied by $e^{at}$ gives equation (\ref{f}):
\begin{equation}
e^{at} \cdot \bigg(dX_t + aX_tdt \bigg) = d(e^{at}X_t)
\end{equation}
Therefore, we can rewrite the SDE as
\begin{equation}
 d(e^{at}X_t) = e^{at} \cdot \bigg(dX_t + aX_tdt \bigg)  = e^{at} \cdot \bigg( amdt+\sigma dB_t \bigg)
\end{equation}
so we obtain
\begin{equation}\label{hi}
 d(e^{at}X_t) = ame^{at}dt+\sigma e^{at} dB_t 
\end{equation}
To solve this SDE, integrate equation (\ref{hi}):
\begin{align}
& \int_0^t d(e^{at}X_t) = \int_0^s  ame^{at}dt+\int_0^s \sigma e^{at} dB_t \\
& X_te^{at}-X_0  = am\bigg(\frac{e^{at}}{a} - \frac{1}{a}\bigg) + \sigma \int_0^s e^{as}dB_s \\
& X_t = m\bigg(1-e^{-at}\bigg) + X_0 e^{-at} + \sigma \int_0^s e^{s-t}dB_s 
\end{align}

Finally, since $X_0=x$:
\begin{equation}\label{xt}
\boxed{X_t = m\bigg(1-e^{-at}\bigg) + x e^{-at} + \sigma \int_0^s e^{s-t}dB_s }
\end{equation}

(ii) We proceed to compute the expectation and variance of $X_t$. Using equation (\ref{xt}):
\begin{align}
\mathbf{E}[X_t] &= \mathbf{E}\left[m(1-e^{-at}) + x e^{-at}  + \sigma \int_0^s e^{s-t}dB_s \right] \\
&= m(1-e^{-at}) + x e^{-at} + \sigma \mathbf{E}\left[ \int_0^s e^{s-t}dB_s \right].
\end{align}

So we need to compute the expectation of the stochastic integral: $I_t =\int_0^s e^{s-t}dB_s $. In the limit $n\longrightarrow \infty$, we can approximate such stochastic integral as a left-hand Riemann sum; for $t_n := t$ :
\begin{equation}
I_t \sim_{n\longrightarrow \infty} \sum_{j=1}^n e^{a(t_j-t)} \bigg(B_{tj}-B_{tj-1}\bigg)
\end{equation}
so the expectation of the stochastic integral is
\begin{align}
\mathbf{E}[I_t] &\sim_{n\longrightarrow \infty} \mathbf{E}\left[\sum_{j=1}^n e^{a(t_j-t)} \bigg(B_{tj}-B_{tj-1}\bigg)\right]  \\
&= \sum_{j=1}^n e^{a(t_j-t)} \mathbf{E}[B_{tj}-B_{tj-1}] = 0
\end{align}
since for a brownian motion, $ \mathbf{E}[(B_{tj}-B_{tj-1}] = 0$. Finally, introducing this result into the expectation of $X_t$ expression, in the limit of $n\longrightarrow \infty$ we get:
\begin{equation}\label{ext}
\boxed{\mathbf{E}[X_t] =m(1-e^{-at}) + x e^{-at} }
\end{equation}

Computing the variance of $X_t$:
\begin{align}\label{final}
Var[X_t] = \mathbf{E}[X_t^2]-(\mathbf{E}[X_t])^2
\end{align}
Where $(\mathbf{E}[X_t])^2$ is, from expression (\ref{ext})
\begin{equation}
(\mathbf{E}[X_t])^2 = \bigg(m(1-e^{-at}) + x e^{-at} \bigg)^2
\end{equation} 
From equation (\ref{xt}) we can compute:
\begin{align}\label{var}
\mathbf{E}[X_t^2] &= \mathbf{E}[(m(1-e^{-at}) + x e^{-at} )^2]+ \mathbf{E}\left[ \sigma^2 (\int_0^s e^{s-t}dB_s)^2 \right] + 2\mathbf{E}\left[(m(1-e^{-at}) + x e^{-at})\cdot \int_0^s e^{s-t}dB_s\right] \\
&= m(1-e^{-at}) + x e^{-at} )^2+ \sigma^2 \mathbf{E}\left[ (\int_0^s e^{s-t}dB_s)^2 \right] + 2(m(1-e^{-at}) + x e^{-at})\mathbf{E}\left[ \int_0^s e^{s-t}dB_s\right]
\end{align}
\begin{itemize}
\item {First, we can compute the second term which is once again, as a left-hand Riemann sum:}
\begin{align}
\mathbf{E}\left[\left(\int_0^s e^{s-t}dB_s\right)^2\right] \sim_{n\longrightarrow \infty} \sum_{j=1}^n e^{2a(t_j-t)} \mathbf{E}[(B_{tj}-B_{tj-1})^2] =  \sum_{j=1}^n e^{2a(t_j-t)} (t_j-t_{j-1})
\end{align}
Where we have used the brownian process property; $\mathbf{E}[(B_{tj}-B_{tj-1})^2] = (t_j-t_{j-1})$. Therefore;  the expectation of a stochastic integral  in the limit of $\longrightarrow \infty$ can be computed:
\begin{equation}
\mathbf{E}\left[\left(\int_0^s e^{s-t}dB_s\right)^2\right] \sim  \int_0^t e^{2a(s-t)ds =  e^{-2at}\left(\frac{e^{2at}}{2a}-\frac{1}{2a}\right)} 
\end{equation}
And so
\begin{equation}
\mathbf{E}\left[\left(\int_0^s e^{s-t}dB_s\right)^2\right]  \sim \frac{1}{2a} (1-e^{2at})
\end{equation}

\item {We now want to compute the third term of the expression (\ref{var}), which is:}
\begin{equation}
2(m(1-e^{-at}) + x e^{-at})\mathbf{E}\left[ \int_0^s e^{s-t}dB_s\right]  = 0
\end{equation}
If we recall that $\mathbf{E}[ \int_0^s e^{s-t}dB_s]  = 0$ from the previous section.
\end{itemize}

Finally, putting these results together, into equation (\ref{final}) we obtain
\begin{equation}
Var[X_t] =  \mathbf{E}[X_t^2]-(\mathbf{E}[X_t])^2 = (m(1-e^{-at}) + x e^{-at} )^2 - (m(1-e^{-at}) + x e^{-at} )^2+ \frac{\sigma^2}{2a} (1-e^{2at}).
\end{equation}
Therefore,
\begin{equation}
\boxed{Var[X_t] =   \frac{\sigma^2}{2a} (1-e^{2at})}
\end{equation}


\pagebreak

\title {{\bf Exercise 3:} {\bf The Cox-Ingersoll-Ross (CIR) model. }}
{\bf The CIR model for the interest rate process $(r_t, t\in[0,T])$ is}:
\begin{align*}
& dr_t = (\alpha - \beta r_t)dt + \sigma \sqrt{r_t}dB_t, && r_0=0
\end{align*} 
{\bf where $\alpha, \beta$ and $\sigma$ are positive constants.} \\

(i) {\bf Applying Ito's formula to the function $f(t,x)=e^{\beta t} x$, compute the expectation of $r_t$. Take the limit of the expectation as} $t\longrightarrow \infty$.\\

(ii) {\bf Using the Euler's scheme on an interval simulate several trajectories of the process for different values of $\alpha, \beta,\sigma$. Take into account that this scheme does not necessarily preserve positivity.} \\

For a function $f(t,x)$, recall Ito's general formula:
\begin{equation}\label{ito}
df(t,x) = \frac{\partial f(x,t)}{\partial t} dt + \frac{\partial f(t,x)}{\partial x}dx + \frac{1}{2}\frac{\partial^2f(x,t)}{\partial x^2}(dx)^2
\end{equation}

Let $f(x,t)$ be given by
\begin{equation}
f(x,t)= e^{\beta t}x
\end{equation}
so
\begin{align}
 \frac{\partial f(x,t)}{\partial t}= \beta e^{\beta t} &&
 \frac{\partial f(t,x)}{\partial x} = e^{\beta t} &&
 \frac{\partial^2f(x,t)}{\partial x^2} = \frac{\partial}{\partial x} (e^{\beta t}) = 0 .
\end{align}

Therefore, from equation (\ref{ito}) we get
\begin{equation}
df(t,x) =  \beta e^{\beta t}dt + e^{\beta t} dx.
\end{equation}
Letting $x := r_t$, the previous equation gives
\begin{equation}\label{dif}
d( e^{\beta t} r_t) =  \beta e^{\beta t}dt + e^{\beta t} dr_t
\end{equation}

The CIR model of interest rates introduces the following differential equation:
\begin{align}
dr_t = (\alpha - \beta r_t)dt + \sigma \sqrt{r_t}dB_t
\end{align} 
which can be rewritten as
\begin{align}\label{cir}
dr_t +\beta r_t dt = \alpha dt + \sigma \sqrt{r_t}dB_t
\end{align} 
Note that the left-hand side of this equation multiplied by $e^{\beta t}$ is the differential of $f(r_t,t)=e^{\beta t}\cdot r_t$:
\begin{equation}
d( e^{\beta t} r_t) =  \beta e^{\beta t}dt + e^{\beta t} dr_t = e^{\beta t} \left(dr_t\right)
\end{equation}
So we can rewrite equation (\ref{cir}) in terms of $d( e^{\beta t} r_t)$ by multiplying by  $e^{\beta t}$ on both sides:
\begin{equation}
d( e^{\beta t} r_t)  = e^{\beta t } \left(dr_t+\beta r_t dt\right) =  e^{\beta t } (\alpha dt + \sigma \sqrt{r_t}dB_t).
\end{equation}
Integrating:
\begin{equation}
d( e^{\beta t} r_t)  =  e^{\beta t } \alpha dt +  e^{\beta t } \sigma \sqrt{r_t}dB_t
\end{equation}
gives
\begin{align}
& \int_0^t d( e^{\beta s} r_s) = \int_0^s e^{\beta s} \alpha ds +  \int_0^s e^{\beta s} \sigma \sqrt{s_t}dB_s \\
& r_t e^{\beta t} -r_0 = \alpha \bigg(\frac{e^{\beta t}}{\beta}-\frac{1}{\beta}\bigg) + \sigma \int_0^s e^(\beta s)\sqrt{r_s}dB_s \\
& r_t = r_0 e^{-\beta t} + \frac{\alpha}{\beta} \bigg(1-e^{-\beta t}\bigg) + \sigma \int_0^s \sqrt{r_s}dB_s 
\end{align}
and since $r_0 = 0$
\begin{equation}
\boxed{r_t = \frac{\alpha}{\beta} \bigg(1-e^{-\beta t}\bigg) + \sigma \int_0^s \sqrt{r_s}dB_s }
\end{equation}

We now proceed to compute the expected value of $r_t$
\begin{align}
\mathbf{E}[r_t] &= \mathbf{E} \left[\frac{\alpha}{\beta} \bigg(1-e^{-\beta t}\bigg) + \sigma \int_0^s \sqrt{r_s}dB_s \right] \\
&= \mathbf{E} \left[\frac{\alpha}{\beta} \bigg(1-e^{-\beta t}\bigg)\right] +\mathbf{E}\left[ \sigma \int_0^s \sqrt{r_s}dB_s \right] \\
&= \frac{\alpha}{\beta} \bigg(1-e^{-\beta t}\bigg) +\sigma \mathbf{E}\left[ \int_0^s \sqrt{r_s}dB_s \right] .
\end{align}

In order to  calculate the expectation of the stochastic integral $I_t = \int_0^s \sqrt{r_s}dB_s$ consider the left-hand Riemann summation:
\begin{align}\label{it}
I_t \sim_{n \rightarrow \infty} \sum_{j=1}^{n} \sqrt{r_{tj-1}}(B_{tj}-B_{tj-1}).
\end{align}
So if we take the expectation of (\ref{it}): 
\begin{equation}
\mathbf{E}[I_t] = \sum_{j=1}^{n} \sqrt{r_{tj-1}}\mathbf{E}[B_{tj}-B_{tj-1}] = 0
\end{equation}
since for a brownian process, $\mathbf{E}[B_{tj}-B_{tj-1}] = 0$.  Finally, 

\begin{equation}
\mathbf{E}[r_t]  = \frac{\alpha}{\beta} \bigg(1-e^{-\beta t}\bigg)
\end{equation}


\end{document}

