\documentclass[12pt]{article}
\usepackage[margin=0.8in]{geometry} 
\usepackage{amsmath,amsthm,amssymb,amsfonts}
\usepackage{graphicx}
\usepackage{algorithmic}
\usepackage{dsfont}
\usepackage{geometry}
\usepackage[toc,page]{appendix}
\usepackage{caption}
\usepackage{subcaption}
\usepackage[utf8]{inputenc}

\newcommand{\N}{\mathbb{N}}
\newcommand{\Z}{\mathbb{Z}}

 
\newenvironment{problem}[2][Problem]{\begin{trivlist}
\item[\hskip \labelsep {\bfseries #1}\hskip \labelsep {\bfseries #2.}]}{\end{trivlist}}
%If you want to title your bold things something different just make another thing exactly like this but replace "problem" with the name of the thing you want, like theorem or lemma or whatever
 
\begin{document}
\title{Pricing Financial Derivatives   \\ Project for Evaluation 2 - February 20, 2017}
\author{Angus McKay, Laura Roman, Euan Dowers, Veronika Kyuchukova}
\date{}
\maketitle

\title {{\bf Exercise 1:} {\bf Black-Scholes model }}
{\bf Consider the Black-Scholes model:}
\begin{align*}
& dX_t = 2X_t dt + 1.5X_tdB_t, && X_0=1
\end{align*}
{\bf (i) Write its solution \\
(ii) Generate on an interval a Brownian path. Using that path plot a trajectory of the solution. Using the same Brownian path use the Euler scheme approximation to plot another trajectory of the equation. Compare these plots with the simulation of the fact solution.} \\

From the differential equation $dX_t = 2X_t dt + 1.5X_tdB_t$ we can rewrite it as follows:
\begin{equation}
\frac{dX_t}{X_t} = 2dt + 1.5dB_t
\end{equation}
So we can integrate as follows:
\begin{equation} \label{uno}
\int_0^t \frac{dX_s}{X_s} = \int_0^t2ds + \int_0^t 1.5dB_s = 2t + 1.5B_t
\end{equation}
Let's now define the following function
\begin{equation}
f(t,X_t)= f(X_t)= ln(X_t)
\end{equation}
Such that its partial derivatives are:
\begin{align}
 \frac{\partial f(x,t)}{\partial t}= 0 &&
 \frac{\partial f(t,x)}{\partial x} = \frac{1}{x}&&
 \frac{\partial^2f(x,t)}{\partial x^2} = \frac{-1}{x^2}  = 0 
\end{align}

Recall Ito's formula:
\begin{align}
df(t,x) = d(lnX_t) =& \frac{\partial f(x,t)}{\partial t} dt + \frac{\partial f(t,x)}{\partial x}dx + \frac{1}{2}\frac{\partial^2f(x,t)}{\partial x^2}(dx)^2= \\
= & \frac{1}{X_t}dX_t +\frac{1}{2}\frac{-1}{X_t^2}(dX_t)^2  
\end{align}
Where $dX_t^2$ can be computed by using the differential equation provided:
\begin{align}
(dX_t)^2   = (2X_tdt+1.5X_tdB_t)^2 = 4X_t^2(dt)^2 + (1.5X_t)^2(dB_t)^2+2\cdot2\cdot1.5X_t^2dtdB_t
\end{align}
From Ito's calculus we do know: $(dt)^2=0; (dB_t)^2=dt$ and $dtdB_t=0$. Followingly;
\begin{align}
(dX_t)^2   = (1.5)^2X_t^2dt
\end{align}
Such that;
\begin{equation}
d(lnX) = \frac{dX_t}{X_t} - \frac{1}{2}\frac{-1}{X_t^2}(1.5 \cdot X_t)^2 dt  
\end{equation}
We can now integrate this expression:
\begin{equation}
\int_0^t d(lnX) = \int_0^t \frac{dX_s}{X_s} - \int_0^t \frac{-1.5^2}{2} ds   
\end{equation}
And finally:
\begin{equation}\label{dos}
\ln \frac{X_t}{X_0} =  \int_0^t \frac{dX_s}{X_s} -  \frac{-1.5^2}{2} t
\end{equation}
Where the term $\int_0^t \frac{dX_s}{X_s}$ from equation (\ref{dos}) has the expression presented in equation (\ref{uno}), such that if substituted we get:
\begin{equation}
\ln \frac{X_t}{X_0} = 1.5B_t + \bigg(2-\frac{1.5^2}{2}\bigg)t
\end{equation}
Therefore;
\begin{equation}
X_t = X_0 \exp\bigg(1.5B_t+0.875\cdot t\bigg)
\end{equation}
Recall $X_0=1$;
\begin{equation}
\boxed{X_t = \exp\bigg(1.5B_t+0.875\cdot t\bigg)}
\end{equation}
\title {{\bf Exercise 2:} {\bf Vasiceck model for interest rates. }}
{\bf Consider the SDE}
\begin{align*}
&dX_t = a(m-X_t)dt + \sigma dB_t; && X_0=x
\end{align*}
{\bf where $a>0, \sigma >0$ and $m\ge 0$. This process tends to drift forward its long-term mean $m$: such a process is called mean-reverting. 
(i) {\bf Differentiating $e^{at}X_t$, find the solution to this SDE.}
(ii) {\bf Compute $\mathbf{E}[X_t]$} and $Var(X_t)$. Take the limit of the expectation as $t\longrightarrow \infty$ to observe the mean-reverting phenomena.}
(iii) {\bf Using the Euler's scheme on an interval simulate several trajectories of the process for different values of $a,\sigma, m$. Do you observe the mean-reverting phenomena? } \\

(i)We are given the following $f(t,X_t)$ function:
\begin{equation}
f(t,X_t)= e^{at}X_t
\end{equation}
We can compute its derivative by:
\begin{equation} \label{f}
d(f(t,X_t)) = d(e^{at}X_t) = e^{at}X_t+X_t\cdot a e^{at}dt = e^{at} \bigg(dX_t + aX_t dt \bigg)
\end{equation}
On the other hand, we are considering the SDE:
\begin{equation}
dX_t = a(m-X_t)dt + \sigma dB_t
\end{equation}
which can be rewritten by reordering its terms as:
\begin{equation}\label{sde}
dX_t + aX_tdt = amdt+\sigma dB_t
\end{equation}
So we realize that the left-hand side of the previous equation (ref{see}) multiplied by $e^{at}$ is in fact equation (\ref{f}):
\begin{equation}
e^{at} \cdot \bigg(dX_t + aX_tdt \bigg) = d(e^{at}X_t)
\end{equation}
Such that we can rewrite the SDE as follows:
\begin{equation}
 d(e^{at}X_t) = e^{at} \cdot \bigg(dX_t + aX_tdt \bigg)  = e^{at} \cdot \bigg( amdt+\sigma dB_t \bigg)
\end{equation}
Therefore we end up with the following equation:
\begin{equation}\label{hi}
 d(e^{at}X_t) = ame^{at}dt+\sigma e^{at} dB_t 
\end{equation}
In order to obtain the solution of such SDE, we integrate equation (\ref{hi}):
\begin{align}
& \int_0^t d(e^{at}X_t) = \int_0^s  ame^{at}dt+\int_0^s \sigma e^{at} dB_t \\
& X_te^{at}-X_0  = am\bigg(\frac{e^{at}}{a} - \frac{1}{a}\bigg) + \sigma \int_0^s e^{as}dB_s \\
& X_t = m\bigg(1-e^{-at}\bigg) + X_0 e^{-at} + \sigma \int_0^s e^{s-t}dB_s 
\end{align}

Finally, recall $X_0=x$ such that:
\begin{equation}\label{xt}
\boxed{X_t = m\bigg(1-e^{-at}\bigg) + x e^{-at} + \sigma \int_0^s e^{s-t}dB_s }
\end{equation}

(ii) We proceed to compute the expectation and variance of $X_t$. Using equation (\ref{xt}):
\begin{align}
\mathbf{E}[X_t] &= \mathbf{E}[m(1-e^{-at}) + x e^{-at} ] + \mathbf{E}[\sigma \int_0^s e^{s-t}dB_s ] = \\
&= m(1-e^{-at}) + x e^{-at} + \sigma \mathbf{E}[ \int_0^s e^{s-t}dB_s ]
\end{align}

So we need to compute the expectation of the stochastic integral: $I_t =\int_0^s e^{s-t}dB_s $. In the limit $n\longrightarrow \infty$, we can approximate such stochastic integral as a left-hand Riemann sum; for $t_n := t$ :
\begin{equation}
I_t \sim_{n\longrightarrow \infty} \sum_{j=1}^n e^{a(t_j-t)} \bigg(B_{tj}-B_{tj-1}\bigg)
\end{equation}
Such that the expectation of the stochastic integral can be computed as follows:
\begin{align}
\mathbf{E}[I_t] &\sim_{n\longrightarrow \infty} \mathbf{E}[\sum_{j=1}^n e^{a(t_j-t)} \bigg(B_{tj}-B_{tj-1}\bigg)]  \\
& \sum_{j=1}^n e^{a(t_j-t)} \mathbf{E}[(B_{tj}-B_{tj-1}] = 0
\end{align}
As for a brownian motion, $ \mathbf{E}[(B_{tj}-B_{tj-1}] = 0$. Finally, introducing this result into the expectation of $X_t$ expression, in the limit of $n\longrightarrow \infty$ we get:
\begin{equation}\label{ext}
\boxed{\mathbf{E}[X_t] =m(1-e^{-at}) + x e^{-at} }
\end{equation}

On the other hand, we now compute the variance of $X_t$:
\begin{align}\label{final}
Var[X_t] = \mathbf{E}[X_t^2]-(\mathbf{E}[X_t])^2
\end{align}
Where $(\mathbf{E}[X_t])^2$ is, from expression (\ref{ext}) is:
\begin{equation}
(\mathbf{E}[X_t])^2 = \bigg(m(1-e^{-at}) + x e^{-at} \bigg)^2
\end{equation} 
From equation (\ref{xt}) we can compute:
\begin{align}\label{var}
\mathbf{E}[X_t^2] &= \mathbf{E}[(m(1-e^{-at}) + x e^{-at} )^2]+ \mathbf{E}[ \sigma^2 (\int_0^s e^{s-t}dB_s)^2 ] + 2\mathbf{E}[(m(1-e^{-at}) + x e^{-at})\cdot \int_0^s e^{s-t}dB_s] = \\
&= m(1-e^{-at}) + x e^{-at} )^2+ \sigma^2 \mathbf{E}[ (\int_0^s e^{s-t}dB_s)^2 ] + 2(m(1-e^{-at}) + x e^{-at})\mathbf{E}[ \int_0^s e^{s-t}dB_s]
\end{align}
\begin{itemize}
\item {Firstly, we can compute the second term which is once again, as a left-hand Rimman sum:}
\begin{align}
\mathbf{E}[(\int_0^s e^{s-t}dB_s)^2] \sim_{n\longrightarrow \infty} \sum_{j=1}^n e^{2a(t_j-t)} \mathbf{E}[(B_{tj}-B_{tj-1})^2] =  \sum_{j=1}^n e^{2a(t_j-t)} (t_j-t_{j-1})
\end{align}
Where we have used the brownian process property; $\mathbf{E}[(B_{tj}-B_{tj-1})^2] = (t_j-t_{j-1})$. Therefore;  the expectation of a stochastic integral  in the limit of $\longrightarrow \infty$ can be computed:
\begin{equation}
\mathbf{E}[(\int_0^s e^{s-t}dB_s)^2] \sim  \int_0^t e^{2a(s-t)ds =  e^{-2at}\bigg(\frac{e^{2at}}{2a}-\frac{1}{2a}\bigg)} 
\end{equation}
And so;
\begin{equation}
\mathbf{E}[(\int_0^s e^{s-t}dB_s)^2]  \sim \frac{1}{2a} (1-e^{2at})
\end{equation}

\item {We now want to compute the third term of the expression (\ref{var}), which is:}
\begin{equation}
2(m(1-e^{-at}) + x e^{-at})\mathbf{E}[ \int_0^s e^{s-t}dB_s]  = 0
\end{equation}
If we recall that $\mathbf{E}[ \int_0^s e^{s-t}dB_s]  = 0$ from the previous section.
\end{itemize}

Finally, putting these results together, into equation (\ref{final}), and :
\begin{equation}
Var[X_t] =  \mathbf{E}[X_t^2]-(\mathbf{E}[X_t])^2 = (m(1-e^{-at}) + x e^{-at} )^2 - (m(1-e^{-at}) + x e^{-at} )^2+ \frac{\sigma^2}{2a} (1-e^{2at})
\end{equation}
Therefore;
\begin{equation}
\boxed{Var[X_t] =   \frac{\sigma^2}{2a} (1-e^{2at})}
\end{equation}


\pagebreak


\title {{\bf Exercise 3:} {\bf The Cox-Ingersoll-Ross (CIR) model. }}
{\bf The CIR model for the interest rate process $(r_t, t\in[0,T])$ is}:
\begin{align*}
& dr_t = (\alpha - \beta r_t)dt + \sigma \sqrt{r_t}dB_t, && r_0=0
\end{align*} 
{\bf where $\alpha, \beta$ and $\sigma$ are positive constants.} \\

(i) {\bf Applying Ito's formula to the function $f(t,x)=e^{\beta t} x$, compute the expectation of $r_t$. Take the limit of the expectation as} $t\longrightarrow \infty$.\\

(ii) {\bf Using the Euler's scheme on an interval simulate several trajectories of the process for different values of $\alpha, \beta,\sigma$. Take into account that this scheme does not necessarily preserve positivity.} \\

For a function $f(t,x)$, recall Ito's general formula as:
\begin{equation}\label{ito}
df(t,x) = \frac{\partial f(x,t)}{\partial t} dt + \frac{\partial f(t,x)}{\partial x}dx + \frac{1}{2}\frac{\partial^2f(x,t)}{\partial x^2}(dx)^2
\end{equation}

Let f(x,t) be:
\begin{equation}
f(x,t)= e^{\beta t}x
\end{equation}
Such that the partial derivatives are:
\begin{align}
 \frac{\partial f(x,t)}{\partial t}= \beta e^{\beta t} &&
 \frac{\partial f(t,x)}{\partial x} = e^{\beta t} &&
 \frac{\partial^2f(x,t)}{\partial x^2} = \frac{\partial}{\partial x} (e^{\beta t}) = 0 
\end{align}

That placed into Ito's general formula (\ref{ito}), we get:
\begin{equation}
df(t,x) =  \beta e^{\beta t}dt + e^{\beta t} dx 
\end{equation}

Let $x := r_t$, the previous equation is rewritten as follows:
\begin{equation} (\label{dif})
d( e^{\beta t} r_t) =  \beta e^{\beta t}dt + e^{\beta t} dr_t
\end{equation}

On the other hand, the CIR model for the interest rate introduces the following differential equation:
\begin{align}
dr_t = (\alpha - \beta r_t)dt + \sigma \sqrt{r_t}dB_t
\end{align} 
Which can be rewritten rearranging terms:
\begin{align}\label{dos}
dr_t +\beta r_t dt = \alpha dt + \sigma \sqrt{r_t}dB_t
\end{align} 
We do realize that the left-hand side of this equation multiplied by $e^{\beta t}$ is in fact the expression of the differential of $f(r_t,t)=e^{\beta t}\cdot r_t$; that is
\begin{equation}
d( e^{\beta t} r_t) =  \beta e^{\beta t}dt + e^{\beta t} dr_t = e^{\beta t} \left(dr_t\right)
\end{equation}
So we can rewrite expression (\ref{dos}) in terms of $d( e^{\beta t} r_t)$ by multiplying  $e^{\beta t}$ to left and right side:
\begin{equation}
d( e^{\beta t} r_t)  = e^(\beta t ) \left(dr_t+\beta r_t dt\right) =  e^(\beta t ) (\alpha dt + \sigma \sqrt{r_t}dB_t)
\end{equation}
Such that we now are left to integrate:
\begin{equation}
d( e^{\beta t} r_t)  =  e^(\beta t ) \alpha dt +  e^(\beta t ) \sigma \sqrt{r_t}dB_t
\end{equation}
That is;
\begin{align}
& \int_0^t d( e^{\beta s} r_s) = \int_0^s e^(\beta s ) \alpha ds +  \int_0^s e^(\beta s ) \sigma \sqrt{s_t}dB_s \\
& r_t e^{\beta t} -r_0 = \alpha \bigg(\frac{e^(\beta t)}{\beta}-\frac{1}{\beta}\bigg) + \sigma \int_0^s e^(\beta s)\sqrt{r_s}dB_s \\
& r_t = r_0 e^(-\beta t) + \frac{\alpha}{\beta} \bigg(1-e^(-\beta t)\bigg) + \sigma \int_0^s \sqrt{r_s}dB_s 
\end{align}
Recall $r_0=0$; therefore:
\begin{equation}
\boxed{r_t = \frac{\alpha}{\beta} \bigg(1-e^{-\beta t}\bigg) + \sigma \int_0^s \sqrt{r_s}dB_s }
\end{equation}

We now proceed to compute the expected value of $r_t$. That is;
\begin{align}
\mathbf{E}[r_t] &= \mathbf{E} [\frac{\alpha}{\beta} \bigg(1-e^(-\beta t)\bigg) + \sigma \int_0^s \sqrt{r_s}dB_s ] = \\
&= \mathbf{E} [\frac{\alpha}{\beta} \bigg(1-e^(-\beta t)\bigg)] +\mathbf{E}[ \sigma \int_0^s \sqrt{r_s}dB_s ] = \\
&= \frac{\alpha}{\beta} \bigg(1-e^(-\beta t)\bigg) +\sigma \mathbf{E}[ \int_0^s \sqrt{r_s}dB_s ] 
\end{align}

We need to calculte the expectation of the stochastic integral $I_t = \int_0^s \sqrt{r_s}dB_s$. In order to do so, we will compute the left-hand Riemann summation; such that:
\begin{align}\label{it}
I_t \sim_{n \rightarrow \infty} \sum_{j=1}^{n} \sqrt{r_{tj-1}}(B_{tj}-B_{tj-1})
\end{align}

So if we take the expectation of (\ref{it}): 
\begin{equation}
\mathbf{E}[I_t] = \sum_{j=1}^{n} \sqrt{r_{tj-1}}\mathbf{E}[B_{tj}-B_{tj-1}] = 0
\end{equation}
As for a brownian process, $\mathbf{E}[B_{tj}-B_{tj-1}] = 0$.  Finally, 

\begin{equation}
\mathbf{E}[r_t]  = \frac{\alpha}{\beta} \bigg(1-e^{-\beta t}\bigg)
\end{equation}


\end{document}

